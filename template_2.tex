\documentclass{article}
\usepackage[a4paper,top=2cm,bottom=2.5cm,left=1.5cm,right=1.5cm,marginparwidth=1.75cm]{geometry}
\usepackage[english]{babel}

%\usepackage[margin=1in]{geometry} 
\usepackage{amsmath,amsthm,amssymb}
\usepackage[colorlinks=true, allcolors=blue]{hyperref}

\usepackage{float}
%% Useful packages
\usepackage{amsmath}
\usepackage[colorinlistoftodos]{todonotes}
\usepackage{listings}
\usepackage{url}
\usepackage{graphicx}
\usepackage{hyperref}
\usepackage{threeparttable}
\usepackage{booktabs}
\usepackage[sc]{mathpazo}
\usepackage{physics}

\newcommand{\R}{\mathbf{R}}  
\newcommand{\Z}{\mathbf{Z}}
\newcommand{\N}{\mathbf{N}}
\newcommand{\Q}{\mathbf{Q}}

\newenvironment{theorem}[2][Theorem]{\begin{trivlist}
\item[\hskip \labelsep {\bfseries #1}\hskip \labelsep {\bfseries #2.}]}{\end{trivlist}}
\newenvironment{lemma}[2][Lemma]{\begin{trivlist}
\item[\hskip \labelsep {\bfseries #1}\hskip \labelsep {\bfseries #2.}]}{\end{trivlist}}
\newenvironment{exercise}[2][Exercise]{\begin{trivlist}
\item[\hskip \labelsep {\bfseries #1}\hskip \labelsep {\bfseries #2.}]}{\end{trivlist}}
\newenvironment{problem}[2][Problem]{\begin{trivlist}
\item[\hskip \labelsep {\bfseries #1}\hskip \labelsep {\bfseries #2.}]}{\end{trivlist}}
\newenvironment{question}[2][Question]{\begin{trivlist}
\item[\hskip \labelsep {\bfseries #1}\hskip \labelsep {\bfseries #2.}]}{\end{trivlist}}
\newenvironment{corollary}[2][Corollary]{\begin{trivlist}
\item[\hskip \labelsep {\bfseries #1}\hskip \labelsep {\bfseries #2.}]}{\end{trivlist}}

\newenvironment{solution}{\begin{proof}[Solution]}{\end{proof}}

\begin{document}

% ------------------------------------------ %
%                 START HERE                  %
% ------------------------------------------ %

\title{\LaTeX\   Template 2} % Replace with appropriate title
\author{BYY\\MGU} % Replace "Author's Name" with your name

\maketitle

\begin{equation}
	\div{\pmb{u}}=0
	\label{ns1}
\end{equation}

\begin{equation}
	\rho(\frac{\partial \pmb{u}}{\partial t}+\pmb{u} \cdot {\grad{\pmb{u}}})=-\grad{p}+\div{(2\mu \vb{D})}+\rho \pmb{g}
	\label{ns2}
\end{equation}

\begin{equation}
	\frac{\partial c}{\partial t}+\div{(c\pmb{u})}=0
	\label{ns3}
\end{equation}	

The volumetric average of density and viscosity:
\begin{equation}
	\rho = c\rho_{mud}+(1-c)\rho_{air}
	\label{c_ave}
\end{equation}
\begin{equation}
	\mu = \frac{1}{\frac{c}{\mu_{mud}}+\frac{1-c}{\mu_{air}}}
	\label{mu_ave}
\end{equation}
where $c$ is the volume fraction of mud ($c\subset [0,1]$). \autoref{mu_ave} is the harmonic  average of viscosity, which works well for large density ratio.

SWEs:
\begin{equation}
	\frac{\partial h}{\partial t}+\frac{\partial q_x}{\partial x}+\frac{\partial q_y}{\partial y}=0
	\label{pl_1}
\end{equation}
\begin{equation}
	\frac{\partial q_x}{\partial t}+\frac{\partial (\beta\frac{q_x^2}{h}+\frac{1}{2}g'h^2)}{\partial x}+\frac{\partial (\beta\frac{q_xq_y}{h})}{\partial y}=gh\sin\theta-\frac{\mu_n}{\rho} (\frac{1+2n}{n }\frac{1}{h})^n(\sqrt{u^2+v^2})^{n-1}u
	\label{pl_2}
\end{equation}
\begin{equation}
	\frac{\partial q_y}{\partial t}+\frac{\partial (\beta\frac{q_yq_x}{h})}{\partial x}+\frac{\partial (\beta\frac{q_y^2}{h}+\frac{1}{2}g'h^2)}{\partial y}=-\frac{\mu_n}{\rho} (\frac{1+2n}{n  }\frac{1}{h})^n(\sqrt{u^2+v^2})^{n-1}v
	\label{pl_3}
\end{equation}

% -----------------------------------------------------
% The following two environments (theorem, proof) are
% where you will enter the statement and proof of your
% first problem for this assignment.
%
% In the theorem environment, you can replace the word
% "theorem" in the \begin and \end commands with
% "exercise", "problem", "lemma", etc., depending on
% what you are submitting. 
% -----------------------------------------------------

\begin{theorem}{(Page 4 $\#$ 6)}
For any metric space $E$, the entire space $E$ is an open set.
\end{theorem}

\begin{proof}
Replace this text with the details of your proof. Mathematical symbols go between dollar signs, like this: $E$, \emph{not} in plain text like this: E. If there are more symbols you want, such as $\in$, you can find them at\\ \url{http://detexify.kirelabs.org/classify.html}. If you want to set expressions or equations out from the text so that they are more readable (great idea!), put them between double dollar signs: $$|x-x_0| < \delta \implies |f(x)-f(x_0)| < \epsilon.$$
The small square marks the end of the proof.
\end{proof}

\cite{Smith:2022qr} here


% ---------------------------------------------------
% Anything after the \end{document} will be ignored by the typesetting.
% ----------------------------------------------------
\bibliographystyle{plain} % We choose the "plain" reference style
\bibliography{sample} % Entries are in the refs.bib file
\end{document}
