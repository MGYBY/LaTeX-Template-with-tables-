\documentclass[a4paper]{article}
\usepackage[a4paper,top=2cm,bottom=2.5cm,left=1.5cm,right=1.5cm,marginparwidth=1.75cm]{geometry}
%% Language and font encodings
\usepackage[english]{babel}
\usepackage[utf8x]{inputenc}
\usepackage{listings}

%% Sets page size and margins

\usepackage{float}
%% Useful packages
\usepackage{amsmath}
\usepackage[colorinlistoftodos]{todonotes}
\usepackage[colorlinks=true, allcolors=blue]{hyperref}
\usepackage{listings}
\usepackage{url}
\usepackage{graphicx}
\usepackage{hyperref}
\usepackage{threeparttable}
\usepackage{booktabs}
\usepackage[sc]{mathpazo}
\usepackage{physics}
\usepackage{pzccal}

\newcommand{\fr}{\rm{Fr}}
\newcommand{\gs}{{\bf{$\mathpzc{Gerris}$}}}
\newcommand{\bs}{{\bf{$\mathpzc{Basilisk}$}}}

\graphicspath{ {./images/} }
% \DeclareGraphicsExtensions{.pdf,.jpg,.png}

%% defined colors
\definecolor{Blue}{rgb}{0,0,0.5}
\definecolor{Green}{rgb}{0,0.75,0.0}
\definecolor{LightGray}{rgb}{0.6,0.6,0.6}
\definecolor{DarkGray}{rgb}{0.3,0.3,0.3}

\title{Roll Wave Simulation Using FOU}
\author{
Boyuan Yu}
\date{\today}


\begin{document}
\maketitle

%\abstract{This is the template for writing reports for the weekly assignments. The page 5 pages. You are expected not to change the font size or style or margin width. As in certain conference 
%submissions, if the template of your submission deviates significantly from this, your submission might incur negative points. So please follow this template and read the instructions.}


\section{Simulation Setup}
\begin{equation}
	\div{\pmb{u}}=0
	\label{ns1}
\end{equation}

\begin{equation}
	\rho(\frac{\partial \pmb{u}}{\partial t}+\pmb{u} \cdot {\grad{\pmb{u}}})=-\grad{p}+\div{(2\mu \vb{D})}+\rho \pmb{g}
	\label{ns2}
\end{equation}

\begin{equation}
	\frac{\partial c}{\partial t}+\div{(c\pmb{u})}=0
	\label{ns3}
\end{equation}	

The volumetric average of density and viscosity:
\begin{equation}
	\rho = c\rho_{mud}+(1-c)\rho_{air}
	\label{c_ave}
\end{equation}
\begin{equation}
	\mu = \frac{1}{\frac{c}{\mu_{mud}}+\frac{1-c}{\mu_{air}}}
	\label{mu_ave}
\end{equation}
where $c$ is the volume fraction of mud ($c\subset [0,1]$). \autoref{mu_ave} is the harmonic  average of viscosity, which works well for large density ratio.

To generate the permanent wave, the sinusoidal disturbance is introduced from the inlet. Following Brock's setup, the simulation cases are listed in \autoref{simulation1}:
\begin{table}[htbp]
	\centering
	\caption{Simulation parameters for sinusoidal inlet disturbance.}
	\label{simulation1}
	\begin{threeparttable}
		\begin{tabular}{ccccccccccc}
			\toprule
			Test & $h_o$(m) & $S_o$ & $\fr$ & $T$ & Dist. Type & $h'/h_o$ & $X$(m) &$N$ &$\rm{Co_n}$&Boundary Condition \\
			\midrule
			Case 1  & 0.00798 & 0.05011 & 3.63 & 0.933 &Sinusoidal & 0.01 & 40 &12000 & 0.08&Inlet-Outlet \\
			Case 2 & 0.00533 & 0.11953 & 5.60 & 1.015 & Sinusoidal & 0.01 & 40 &12000 &0.08&Inlet-Outlet\\
			\bottomrule    
		\end{tabular}

	\end{threeparttable}
\end{table}

The wave coarsening happens when there exist more than one sinusoidal component in the inlet disturbance. Therefore, one inlet disturbance is set up as:
\begin{equation}
\label{cd}
\begin{split}
h=h_o[1+\epsilon(\sum_{T_i}\sin(\frac{2\pi}{T_i}  t))]
\end{split}
\end{equation}
where $T_i$ is selected as $0.6,0.9,1.2,1.5,1.8$ s, namely the disturbance with 5 sinusoidal components. The setup for investigating the coarsening   is listed in \autoref{simulation2}.
\begin{table}[htbp]
	\centering
	\caption{Simulation parameters for investigating coarsening effect.}
	\label{simulation2}
	\begin{threeparttable}
		\begin{tabular}{ccccccccccc}
			\toprule
			Test & $h_o$(m) & $S_o$ & $\fr$ & $T$ & Dist. Type & $h'/h_o$ & $X$(m) &$N$&$\rm{Co_n}$&Boundary Condition \\
			\midrule
			Case 3  & 0.00798 & 0.05011 & 3.63 & -- &\autoref{cd} & 0.01 & 400 &40000&0.08&Inlet-Outlet\\
			\bottomrule    
		\end{tabular}

	\end{threeparttable}
\end{table}

The case with square disturbance is listed in \autoref{simulation3} to make comparison with the sinusoidal disturbance (case 1).
\begin{table}[htbp]
	\centering
	\caption{Simulation parameters for inlet square disturbance.}
	\label{simulation3}
	\begin{threeparttable}
		\begin{tabular}{cccccccccccc}
			\toprule
			Test & $h_o$(m) & $S_o$ & $\fr$ & $T$ & Dist. Type & $h'/h_o$ & $X$(m) &$N$&$\rm{Co_n}$&Boundary Condition\\
			\midrule
			Case 4  & 0.00798 & 0.05011 & 3.63 & 0.933 &Square & 0.01 & 40 &12000&0.08&Inlet-Outlet\\
			\bottomrule    
		\end{tabular}

	\end{threeparttable}
\end{table}

\section{Numerical Results}
The convection term is discretized using first order upwind scheme (FOU) for all the cases. Courant number is selected sufficiently small to avoid numerical instability.
\subsection{Sinusoidal Disturbance}
The water surface profiles for case 1 and case 2 are shown in \autoref{case1t200} and \autoref{case2t200}. The comparison between Brock's results (experimental and theoretical) and numerical results are shown in \autoref{case12comp}.

\subsection{Coarsening Effect}
To observe the wave coarsening phenomenon, the simulation time is sufficiently long (500 s) and channel length is also set to be sufficiently long(400 m). At the time moment $t=500$s, the surface profile at $x=50-100$m and $x=350-400$m  is shown in \autoref{case3}. It can be observed that the wave combining and overtaking occurs.

\subsection{Square Disturbance}
The water surface profile for case 4 at $t=200$s is shown in \autoref{case4}. Through the comparison between \autoref{case1t200} and \autoref{case4}, it can be concluded that the square and sinusoidal inlet disturbance with the same period generate virtually the same roll wave at the downstream.

%\begin{figure}[htbp]
%\centering
%\includegraphics[width=.3\textwidth]{fig2.jpg}
%\caption{This figure is an example of a figure caption taking more than one
%  line and justified considering margins mentioned in Section~\ref{sec:figs}.}
%\label{fig:exampleFig2}
%\end{figure}

\bibliographystyle{plain} % We choose the "plain" reference style
\bibliography{bibfile} % Entries are in the refs.bib file

\end{document}
